\begin{document}
\section{Introduzione}

Nell'ambito dell'Ingegneria del Software, il termine \emph{testing} denota il processo di analisi di un sistema software avente lo scopo di individuare errori logici a run-time e di verificare la correttezza del sistema rispetto alle specifiche iniziali concordate con il committente.\\
Tale processo si colloca all'interno del ciclo di vita del software, e ne costituisce una fase fondamentale: un testing ben strutturato permette quasi sempre un beneficio non trascurabile in termini di risparmio di tempo in fase manutentiva, e di conseguenza anche in termini di spese sostenute.\\
Avere strumenti in grado di automatizzare questo processo è spesso una scelta vincente, sia per quanto concerne l'affidabilità dei test generati, sia in termini di rapidità con cui il processo si svolge.\\
Scopo di questo lavoro di tesi è quello di realizzare uno strumento software atto ad automatizzare parte del processo di testing per la libreria Java JSetL.\\
Sebbene siano molti gli strumenti ad oggi disponibili per effettuare testing automatico (si veda a tal proposito [5]), in questo lavoro di tesi si è scelto di realizzare uno strumento ex novo, specifico per la libreria JSetL.\\

JSetL è una libreria Java sviluppata presso il Dipartimento di Matematica e Informatica dell’Università di Parma.\\
La libreria nasce con l'obiettivo di combinare il paradigma object-oriented di Java con alcuni dei concetti classici del paradigma logico a vincoli, tra i quali variabili logiche, unificazione, risoluzione di vincoli e non determinismo (tutti i dettagli sono reperibili in [1] e [2]).\\
Lo strumento software realizzato in questo lavoro di tesi si focalizza sul testing dei vincoli offerti dalla libreria JSetL, analizzandone sia aspetti di correttezza semantica, sia aspetti di efficienza.\\
L'approccio utilizzato per la realizzazione dello strumento si basa sulla creazione di due generatori automatici di casi di test, i quali forniscono in output una casistica completa di test sulla base di precise specifiche di input inserite dall'utente.
\clearpage

L'elaborato di tesi è così strutturato:
\begin{itemize}
    \item \textbf{Capitolo 1}: In questo capitolo verrà brevemente illustrato il concetto di testing, classificandone le varie tipologie, individuandone le diverse fasi che ne compongono il processo, ed infine analizzando pro e contro del testing manuale e di quello automatizzato (si veda a tal proposito [3]).
    
    \item \textbf{Capitolo 2}: In questo capitolo verrà presentato l'ambiente in cui si colloca lo strumento software realizzato: la libreria JSetL, il linguaggio CLP(\(\mathcal{SET}\))/\mathcal{L\textsubscript{\(\mathcal{BR}\)}}, l'interprete \{log\} ed infine il framework JUnit (per approfondimenti, si rimanda a [4]).
    
    \item \textbf{Capitolo 3}: In questo capitolo verrà presentata l'architettura dello strumento software realizzato per questo lavoro di tesi, includendo le motivazioni che hanno portato al suo sviluppo.

    \item \textbf{Capitolo 4}: In questo capitolo verrà analizzato nel dettaglio lo strumento software realizzato, descrivendo sintassi e semantica dei diversi file di input/output e degli script di compilazione. Verranno infine presentati i due programmi generatori scritti in linguaggio C++.
    
    \item \textbf{Capitolo 5}: In questo capitolo verranno analizzati 12 test set per JSetL prodotti tramite lo strumento realizzato, fornendo una dimostrazione pratica di utilizzo dello stesso.\\
    Verranno infine fatte alcune analisi sui tempi di esecuzione rilevati da un campione di 5 test set.
    
\item \textbf{Capitolo 6}: In quest'ultimo capitolo verranno tratte alcune conclusioni sul lavoro di tesi, analizzando i punti di forza dello strumento realizzato ed in particolare in che modo lo strumento potrà fornire supporto al continuo sviluppo della libreria JSetL.\\
Infine, verrà lasciato spazio ad alcuni spunti di rilievo in merito ad eventuali lavori futuri e miglioramenti apportabili allo strumento realizzato.

\end{itemize}
\end{document}
