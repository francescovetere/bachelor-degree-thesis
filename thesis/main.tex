\documentclass[12pt,a4paper,oneside,openright]{article}
\linespread{1.3}
\usepackage{amsmath}
\usepackage{empheq}
\usepackage[theorems,skins,most]{tcolorbox}
\usepackage{mathtools}
\usepackage{bm}
\bibliographystyle{plain}
\usepackage[options]{natbib}

\newtcolorbox{mymathbox}[1][]{colback=white, sharp corners, #1}
%\newtcolorbox{mymathbox}[1][]{colback=white, sharp corners}
\usepackage{lipsum} % for creating dummy text
%\usepackage{tcolorbox}
\usepackage{hyperref}
\usepackage{parskip}
\usepackage[utf8]{inputenc}
\usepackage{graphicx}
%\usepackage{import}
\usepackage[italian]{babel}
\usepackage{subfiles}
\usepackage{fancyhdr}
\usepackage{algorithm}
%\usepackage[noend]{algpseudocode}
%\usepackage[linesnumbered, ruled]{algorithm2e}
\usepackage{algpseudocode}
\usepackage{listings}
 \usepackage{natbib}
 \usepackage[dvipsnames]{xcolor}
 
\definecolor{arsenic}{rgb}{0.35, 0.35, 0.35}
 \lstset{
 frame=bt,
 %frameround=tttt,
%mathescape=true,
  %language=Java,
  breaklines=true,
  showstringspaces=false,
  columns=flexible,
  numbers=none,
  %commentstyle=\color{MidnightBlue},
 %stringstyle=\color{gray},
  %stringstyle=\color{purple},
  basicstyle=\footnotesize\ttfamily,
  %literate=*{\$}{{\textcolor{arsenic}{\$}}}{1},
  tabsize=4
}

\usepackage{geometry}

 % Package aggiunti
  \usepackage{amssymb}				% matematica
  \usepackage{amsmath}				% matematica
  \usepackage{amsthm}				% matematica -> stile teoremi, def, proposizioni
  \usepackage{amsbsy}				% for bold math symbol
  \usepackage{cases}				% sistemi di equazioni con numerazione e sottonumerazione
  \usepackage{booktabs}				% tabelle con toprule ecc
  \usepackage{textcomp}				% per il simbolo di gradi
  \usepackage{subfig}
  \usepackage{scalerel}             % per riscalare i sotto item
  \usepackage{dirtytalk}			% per usare /say{"..."}, ovvero "..."

\newcommand\Blacksquare{\scaleobj{0.6}{\blacksquare}} %riscalo per i sotto item
\newcommand\Blacktriangleright{\scaleobj{0.9}{\blacktriangleright}} %riscalo per i sotto sotto item

\renewcommand*\contentsname{Summary}
\begin{document}

\newgeometry{
	paper=a4paper, % Change to letterpaper for US letter
	inner=2.5cm, % Inner margin
	outer=3.8cm, % Outer margin
	bindingoffset=.5cm, % Binding offset
	top=1.5cm, % Top margin
	bottom=1.5cm, % Bottom margin
	%showframe, % Uncomment to show how the type block is set on the page
}

\begin{figure}

    \centering
    \includegraphics[scale=0.2]{images/logo_unipr.png}
    \label{fig:my_label}
\end{figure}



\begin{titlepage}
    
   \begin{center}
      \LARGE\texttm{Dipartimento di Scienze \\Matematiche, Fisiche e Informatiche}\\
       \vspace*{\stretch{0.25}}
      \Large\texttm{Corso di Laurea in Informatica}\\
      \vspace*{\stretch{1.5}}
      \LARGE\textbf{
    	Generazione automatica ed esecuzione di casi di test per la libreria Java JSetL
    	}
      \vspace*{\stretch{2.0}}
      
      
      \large\textbf{
      	Automatic generation and execution of test cases for the Java library JSetL
      	}
      \vspace*{\stretch{2.0}}
    \end{center}
    \\[3cm]
   
    \begin{minipage}[t]{0.5\textwidth}
\begin{flushleft}
\Large{Relatore:} \\
            \large\textbf{Prof. Gianfranco Rossi}
\end{flushleft}
\end{minipage}
\hfill
\begin{minipage}[t]{0.5\textwidth}
\begin{flushright}
\Large{Candidato:} \\
            \large\textbf{Francesco Vetere}
\end{flushright}
\end{minipage}\\

\begin{center}
        \vspace*{\stretch{1.0}}
        \Large\texttm{Anno Accademico 2018/2019}
\end{center}
    
\end{titlepage}

\newpage
\thispagestyle{empty}
\mbox{}

\restoregeometry
\clearpage
\pagestyle{playn}
\begin{flushright}
\begin{dedication}
%\textit{$ \langle $ dedica $ \rangle $}
\textit{Ai miei genitori, Giusy e Domenico.\\A mia sorella, Giulia.}
\end{dedication}
\end{flushright}

\newpage
\thispagestyle{empty}
\mbox{}

\clearpage

\tableofcontents

\pagestyle{fancy}
\headheight 15pt
\renewcommand{\chaptermark}[1]{\markboth{{\chaptername}\ \thechapter.\hspace{1em}#1}{}}
\renewcommand{\footrulewidth}{0pt}


%\lhead{}
%\rhead[\fancyplain{}{}]{\fancyplain{}{\leftmark}} 
%\chead{}
%\lfoot{}
%\cfoot{\thepage}
%\rfoot{}



\clearpage
\setlength{\parindent}{0in}

\setcounter{secnumdepth}{0} % numero solo le section
\subfile{introduzione}
\setcounter{secnumdepth}{3} % numero fino alle subsubsection

\clearpage 
\subfile{chapter1/chapter1_main}

\clearpage 
\subfile{chapter2/chapter2_main}

\clearpage
\subfile{chapter3/chapter3_main}

\clearpage
\subfile{chapter4/chapter4_main}

\clearpage
\subfile{chapter5/chapter5_main}

\clearpage
\subfile{chapter6/chapter6_main}

\setcounter{secnumdepth}{0} % numero solo le section
\clearpage
\subfile{ringraziamenti}
\setcounter{secnumdepth}{3} % numero fino alle subsubsection

\clearpage
%\addcontentsline{toc}{section}{Riferimenti bibliografici}
\subfile{bibliography}
\addcontentsline{toc}{section}{Riferimenti bibliografici}

\end{document}
