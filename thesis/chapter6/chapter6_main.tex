\begin{document}
\section{Conclusioni e lavori futuri}

In questo lavoro di tesi è stato affrontato il problema del testing dei vincoli della libreria Java JSetL.
Il problema è stato affrontato realizzando uno strumento software ad hoc avente il compito di generare i test cases in maniera automatica.
La realizzazione dello strumento ha quindi permesso di avere a disposizione un tool dalla duplice funzione.\\
Un primo aspetto utile è dato dalla possibilità di testare la correttezza di qualsiasi vincolo presente nella libreria (anche eventualmente di vincoli che verranno sviluppati in futuro e quindi non esistenti al momento della realizzazione dello strumento).\\
Un secondo aspetto utile è dato dalla possibilità di effettuare dei benchmark sistematici sull'esecuzione dei test set generati. 
Ciò permette di evidenziare quali vincoli riescano a scalare meglio all'aumentare delle dimensioni dei parametri: i dati ricavati dall'analisi di questi tempi forniscono un valido suggerimento che può indicare dove sia più opportuno iniziare a lavorare nel caso si voglia rendere la libreria nel suo complesso sempre più efficiente.\\ 

Per quanto riguarda eventuali sviluppi futuri, vi sono almeno due aspetti non trattati da questa tesi su cui si potrebbe lavorare.\\
Sicuramente il multi-threading è ancora tutto da esplorare: lo strumento genera al momento classi Java che sfruttano l'approccio di default di JUnit 4, ossia quello di un unico thread master il quale esegue sequenzialmente i diversi metodi di test.\\
A partire dalla versione 5.3 di JUnit tuttavia, è possibile eseguire più metodi di test in parallelo tra loro (sebbene la feature, al momento della scrittura di questa tesi, sia ancora in fase sperimentale).\\
Si potrebbe quindi fare un confronto tra esecuzione sequenziale e parallela, capire se i vantaggi sono realmente tangibili oppure trascurabili, e preoccuparsi delle problematiche di data race tipiche di un approccio parallelo.\\

Un ulteriore miglioramento possibile potrebbe consistere nella creazione di un tool che sia in grado di visualizzare graficamente i tempi di esecuzione.\\
Questo ipotetico programma, ricevuto in input il file \texttt{times.txt}, potrebbe generare un istogramma in maniera automatica e, in base a soglie temporali preimpostate dall'utente, generare un report in cui vengano descritti quali sono quei vincoli che fungono da collo di bottiglia per il test set.

\end{document}
