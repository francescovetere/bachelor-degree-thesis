\begin{document}

\section{Descrizione dell'ambiente}

In questo capitolo viene presentato l'ambiente in cui opera lo strumento software realizzato per questo lavoro di tesi.
\subsection{JSetL}

\subsubsection{Principali features}
JSetL è una libreria Java sviluppata presso il Dipartimento di Matematica e Informatica dell'Università di Parma.\\
Il suo scopo è quello di combinare il paradigma object-oriented di Java con alcuni concetti classici del paradigma logico a vincoli, tra i quali:\\

\begin{itemize}
\item \textbf{Variabili logiche}\\ Sono le variabili tipiche del paradigma logico-funzionale.\\
Possono essere inizializzate o non inizializzate, e il loro valore può potenzialmente essere di ogni tipo ammesso dal linguaggio: può in particolare essere determinato come risultato della risoluzione di vincoli che coinvolgono le variabili stesse.\\

\item \textbf{Unificazione}\\ Il meccanismo dell'unificazione è fondamentale in questo paradigma di programmazione, e permette sia di testare l'uguaglianza tra oggetti, sia di assegnare valori alle variabili logiche non inizializzate eventualmente contenute al loro interno nel tentativo di rendere uguali i due termini del confronto.\\

\clearpage

\item \textbf{Risoluzione di vincoli}\\ JSetL fornisce una serie di vincoli di base quali uguaglianza, disuguaglianza, operazioni insiemistiche di
base come differenza, unione, intersezione e molte altre.\\ 
Un vincolo viene prima aggiunto allo store dei vincoli (\emph{constraint store}), e successivamente viene risolto tramite un risolutore (\emph{solver}) attraverso regole di riscrittura sintattica.\\
JSetL permette inoltre la creazione di vincoli composti (tramite congiunzione, disgiunzione e implicazione) e vincoli definiti da utente.\\

\item \textbf{Non determinismo}\\ I vincoli in JSetL vengono risolti in maniera non deterministica: sia perché l'ordine di risoluzione degli stessi non è rilevante, sia perché nel processo di risoluzione si fa ampio di uso di choice-points e di backtracking.\\

\end{itemize}

Inoltre, JSetL fornisce \textbf{insiemi} e \textbf{liste} come strutture dati fondamentali, su cui è possibile operare tramite una serie di vincoli, in particolare vincoli insiemistici che implementano le classiche operazioni su insiemi, quali unione, intersezione, appartenenza, ecc...\\ La principale differenza tra liste e insiemi risiede nel fatto che nelle liste sono rilevanti ordine e ripetizioni degli elementi, negli insiemi no.\\
Entrambe le strutture dati permettono di collezionare elementi di tipo generico: in particolare, esse possono essere parzialmente specificate (ovvero possono contenere variabili logiche non inizializzate, aspetto molto importante in JSetL).\\

Vengono di seguito brevemente analizzate le classi principali che compongono la libreria e che compariranno di frequente in questa tesi.\\

\subsubsection{Classi fondamentali}

\textbf{LVar}\\
La classe \texttt{LVar} implementa il concetto di variabile logica descritto in precedenza.\\
Oggetti di classe \texttt{LVar} possono essere creati associandovi un valore (\emph{bound}), lasciandoli non inizializzati (\emph{unbound}), e possono o meno avere un nome identificativo associato.\\
Esempi:\\
\begin{lstlisting}
// Creazione di una LVar unbound senza nome associato
LVar A = new LVar();

//Creazione di una LVar bound senza nome associato
LVar B = new LVar(1);

//Creazione di una LVar bound con nome associato
LVar C = new LVar("C", 3);
\end{lstlisting}

\textbf{LSet}\\
La classe \texttt{LSet} implementa il concetto di insieme logico.\\
Come visto in precedenza, un insieme logico può contenere qualsiasi oggetto (in particolare anche un altro \texttt{LSet}), ignorando ordine e ripetizione degli elementi.\\
Similmente alle \texttt{LVar}, oggetti di tipo \texttt{LSet} possono essere bound o unbound, con o senza nome identificativo associato, e possono inoltre essere completamente specificati (\emph{chiusi}) oppure parzialmente specificati (\emph{aperti}).\\
Esempi:\\

\begin{lstlisting}
/* Creazione di un LSet unbound senza nome associato
   (rappresenta un insieme completamente variabile) */
LSet A = new LSet();

/* Creazione di un LSet completamente specificato, bound e senza nome associato
   (rappresenta l'insieme {1,2,3}) */
LSet B = LSet.empty().ins(1).ins(2).ins(3);

/* Creazione di un LSet parzialmente specificato, unbound e con nome associato
   (rappresenta l'insieme {_LV1, _LV2 | C}) */
LSet C = new LSet("C").ins(new LVar()).ins(new LVar());
\end{lstlisting}

Come si vedrà in seguito, recentemente è stata aggiunta a JSetL una nuova importante categoria di insiemi logici, quella dei Restricted Intensional Set (si veda [6]), che verranno analizzati e testati nel capitolo 5.9.\\
Grazie ad un recente lavoro di tesi inoltre (si veda [7]), JSetL implementa, in aggiunta al linguaggio CLP(\(\mathcal{SET}\)) che verrà brevemente discusso nel capitolo 2.2, anche il linguaggio \mathcal{L\textsubscript{\(\mathcal{BR}\)}}, ossia l'estensione di CLP(\(\mathcal{SET}\)) con il concetto di relazione binaria.\\
Questa estensione comporta l'aggiunta di classi come \texttt{LRel} e \texttt{LMap}, di seguito descritte.\\

\textbf{LRel}\\
La classe \texttt{LRel} implementa le relazioni binarie in JSetL.\\
Una relazione binaria è rappresentata come un particolare insieme i cui elementi sono coppie ordinate: per questo motivo la classe \texttt{LRel} estende la classe \texttt{LSet}.\\
Un'istanza di \texttt{LRel} è dunque un'istanza di \texttt{LSet} in cui gli elementi sono coppie logiche, rappresentate dalla classe \texttt{LPair}.\\
Esempi:\\

\begin{lstlisting}
/* Creazione di una LRel unbound con nome associato "A" */
LRel A = new LRel("A");

/* Creazione della LRel {[0,1],[2,3]}, senza nome associato */
LRel B = LRel.empty().ins(new LPair(0, 1)).ins(new LPair(2, 3));
\end{lstlisting}

\clearpage

\textbf{La classe LMap}\\
La classe \texttt{LMap} implementa le funzioni parziali in JSetL.\\
Una funzione parziale è rappresentata come una particolare relazione binaria in cui i primi componenti di tutte le coppie della relazione sono diversi gli uni dagli altri: per questo motivo la classe \texttt{LMap} estende la classe \texttt{LRel}.\\
Esempi:\\

\begin{lstlisting}
/* Creazione di una LMap unbound con nome associato "A" */
LMap A = new LMap("A");

/* Creazione della LMap {[1,'a'],[2,'b']}, senza nome associato */
LMap B = LMap.empty().ins(new LPair(1, 'a')).ins(new LPair(2, 'b'));
\end{lstlisting}

\textbf{La classe Constraint}\\
La classe \texttt{Constraint} rappresenta la nozione di vincolo in JSetL.\\
Un vincolo è un'operazione che può essere applicata ad una variabile logica (\texttt{LVar}) o ad un insieme logico (\texttt{LSet}, \texttt{LRel}, \texttt{LMap}).\\
Un vincolo può essere atomico o composto.\\

\begin{itemize}
\item \textbf{Vincolo atomico}\\
Un vincolo atomico può essere di queste tipologie:
\begin{itemize}
\item Vincolo \emph{vuoto}, denotato da \texttt{[]}
\item $ e_{0}.op(e_{1},...,e_{n}) $ oppure $op(e_{0},e_{1},...,e_{n}) $ con $ n = 0,...,3 $\\
dove $op$ è il nome del vincolo, e $e_{i} (0 \le i \le 3)$ sono espressioni il cui tipo dipende da $op$.\\
\end{itemize}

\clearpage

\item \textbf{Vincolo composto}\\
Un vincolo composto può essere ottenuto tramite:
\begin{itemize}
\item Congiunzione ($ c_{1}.and(c_{2}) $)
\item Disgiunzione ($ c_{1}.or(c_{2}) $)
\item Implicazione ($ c_{1}.impliesTest(c_{2}) $)\\
\end{itemize}

\end{itemize}

Vengono di seguito riportate due tabelle con i vincoli principali per \texttt{LSet} e \texttt{LRel}.\\

\textsc{Vincoli di LSet}
\begin{table}[h]
%\centering
\begin{tabular}{|l|l|l|}
\hline
\textsc{Vincolo} & \textsc{Sintassi}                	& \textsc{Significato}                               	\\ \hline
\textbf{eq}      & Constraint eq(LSet s)            	& \texttt{this} $=$ \texttt{s}                      	\\ \hline
\textbf{disj}    & Constraint disj(LSet s)          	& \texttt{this} $\cap$ \texttt{s} $=$ $\varnothing$ 	\\ \hline
\textbf{union}   & Constraint union(LSet s, LSet q) 	& \texttt{q} $=$ \texttt{this} $\cup$ \texttt{s}     	\\ \hline
\textbf{in}      & Constraint in(LSet s)            	& \texttt{this} $\in$ \texttt{s}                    	\\ \hline
\textbf{diff}    & Constraint diff(LSet s, LSet q)  	& \texttt{q} $=$ \texttt{this} $\setminus$ \texttt{s}   \\ \hline
\textbf{inters}  & Constraint inters(LSet s, LSet q)	& \texttt{q} $=$ \texttt{this} $\cap$ \texttt{s}        \\ \hline
\textbf{subset}  & Constraint subset(LSet s)        	& \texttt{this} $\subseteq$ \texttt{s}                  \\ \hline
\textbf{size}    & Constraint size(IntLVar n)       	& \texttt{n} $=$ $|$ \texttt{this} $|$              	\\ \hline
\end{tabular}
\end{table}


\textsc{Vincoli di LRel}

\begin{table}[h]
%\centering
\begin{tabular}{|l|l|l|}
\hline
\textsc{Vincolo} & \textsc{Sintassi}                & \textsc{Significato}                               	\\ \hline
\textbf{id}      & Constraint id(LSet a)		    & $this = \{(x, x) \:| \: x \in a \}$  	\\ \hline
\textbf{inv}     & Constraint inv(LRel s)           & $s = \{ (y, x) \:| \: (x, y) \in this \}$ 	\\ \hline
\textbf{comp}    & Constraint comp(LRel s, LRel q)  & $q = \{ (x, z) \:| \: \exists y : (x, y) \in this \: \wedge $\\ & & $\: (y, z) \in s \}$ \\ \hline

\textbf{dom}     & Constraint dom(LSet a)           & $a = \{ x \:| \: \exists y : ((x, y) \in this)\}$ 	\\ \hline
\textbf{ran}     & Constraint ran(LSet a)           & $a = \{ y \:| \: \exists x : ((x, y) \in this)\}$ 	\\ \hline
\end{tabular}
\end{table}

\clearpage

\textbf{La classe Solver}\\
Il risolutore di vincoli è implementato dalla classe \texttt{Solver}.\\
La classe fornisce metodi per aggiungere, mostrare e risolvere vincoli.\\
I vincoli vengono risolti tramite regole di riscrittura sintattica, e la procedura di risoluzione termina quando non vi sono più regole da applicare (oppure nel momento in cui viene lanciata un'eccezione di tipo \texttt{Failure}, nel caso in cui il vincolo non sia soddisfacibile).\\
I metodi principali della classe sono i seguenti:\\
\begin{itemize}
\item public void \textbf{add}(Constraint c)\\ 
Aggiunge il vincolo \texttt{c} al constraint store del risolutore.\\
\item public void \textbf{showStore}()\\ 
Stampa la congiunzione di tutti i vincoli presenti nel constraint store del risolutore ancora in forma non risolta.\\
\item public void \textbf{solve}()\\ 
Risolve i vincoli nel constraint store del risolutore: se la risoluzione non è possibile, solleva un'eccezione di tipo \texttt{Failure}.\\
\item public boolean \textbf{check}()\\ 
Analogo al metodo \texttt{solve()}, con la differenza che non vengono sollevate eccezioni: ritorna \texttt{true} se i vincoli presenti nel constraint store del risolutore sono soddisfacibili, \texttt{false} altrimenti.\\
\end{itemize}

\end{itemize}

\clearpage

\subsection{CLP(\(\mathcal{SET}\)) e \{log\}}
La programmazione con vincoli (\emph{CP, Constraint Programming}) può essere vista come una forma di programmazione dichiarativa che permette di manipolare esplicitamente vincoli (ossia relazioni su opportuni domini).\\
In particolare, se il linguaggio utilizzato per la programmazione con vincoli è di tipo logico si parla di programmazione logica con vincoli (\emph{CLP, Constraint Logic Programming}).\\
Un esempio di linguaggio di programmazione logica con vincoli è CLP(\(\mathcal{SET}\)), in cui il dominio dei vincoli è quello degli insiemi logici.
Come visto in precedenza nel capitolo 2.1.1 inoltre, l'estensione del linguaggio a vincoli presente in CLP(\(\mathcal{SET}\)) con l'aggiunta delle relazioni binarie prende il nome di linguaggio \mathcal{L\textsubscript{\(\mathcal{BR}\)}}.\\

\mathcal{L\textsubscript{\(\mathcal{BR}\)}} è implementato nella libreria JSetL, in un contesto di programmazione O-O.\\
Esiste un'implementazione di \mathcal{L\textsubscript{\(\mathcal{BR}\)}} anche per Prolog, che prende il nome di \{log\} (da leggersi \emph{setlog}).\\

\{log\} è un interprete inizialmente sviluppato presso il Dipartimento di Matematica e Informatica di Udine.\\
Il suo scopo è quello di implementare il linguaggio CLP(\(\mathcal{SET}\)) e le successive estensioni del suo linguaggio a vincoli in un contesto di programmazione logica.\\
\{log\} è di fatto molto simile a Prolog nella sintassi, ed è utilizzabile sia in modalità interattiva a riga di comando, sia come modulo esterno da includere nel proprio sorgente Prolog.\\

\clearpage

Anche \{log\} fornisce una serie di vincoli di base su insiemi e relazioni: di seguito viene proposta una panoramica dei vincoli visti in precedenza per il caso di JSetL.\\\\

\begin{table}[h]
\textsc{Vincoli su insiemi}\\

%\centering
\begin{tabular}{|l|l|l|}
\hline
\textsc{Vincolo} & \textsc{Sintassi}                	& \textsc{Significato}                               	\\ \hline
\textbf{=}       & A = B                             	& \texttt{A} $=$ \texttt{B}                      	\\ \hline
\textbf{disj}    & disj(A, B)                        	& \texttt{A} $\cap$ \texttt{B} $=$ $\varnothing$ 	\\ \hline
\textbf{un}      & un(A, B, C) 	                        & \texttt{C} $=$ \texttt{A} $\cup$ \texttt{B}	  	\\ \hline
\textbf{in}      & x in A                           	& \texttt{x} $\in$ \texttt{A}                    	\\ \hline
\textbf{diff}    & diff(A, B, C)                    	& \texttt{C} $=$ \texttt{A} $\setminus$ \texttt{B}   \\ \hline
\textbf{inters}  & inters(A, B, C)                  	& \texttt{C} $=$ \texttt{A} $\cap$ \texttt{B}        \\ \hline
\textbf{subset}  & subset(A, B) 				       	& \texttt{A} $\subseteq$ \texttt{B}                  \\ \hline
\textbf{size}    & size(A, N)					       	& \texttt{N} $=$ $|$ \texttt{A} $|$              	\\ \hline
\end{tabular}
\end{table}




\begin{table}[h]
\newline
\vspace*{1 cm}
\newline
\textsc{Vincoli su relazioni}\\

%\centering
\begin{tabular}{|l|l|l|}
\hline
\textsc{Vincolo} & \textsc{Sintassi}              	    & \textsc{Significato}                               	\\ \hline
\textbf{id}      & id(R, A)							    & \texttt{R} $= \{(x, x) \:| \: x \in A \}$  	\\ \hline
\textbf{inv}     & inv(R, S)           					& \texttt{R} $=$ \texttt{S}^{-1} 	\\ \hline
\textbf{comp}    & comp(R, S, T)                        & \texttt{T} $=$ \texttt{R} $\circ$ \texttt{S}	\\ \hline
\textbf{dom}     & dom(R, A)       					    & \texttt{dom R} $=$ \texttt{A} 	\\ \hline
\textbf{ran}     & ran(R, A)          					& \texttt{ran R} $=$ \texttt{A}	\\ \hline
\end{tabular}
\end{table}

\clearpage

\subsection{JUnit}
JUnit è un framework open-source per il linguaggio di programmazione Java pensato per fare unit testing in maniera semplice ed organizzata.\\
Il framework si basa principalmente su annotazioni e asserzioni, utilizzate in maniera opportuna.\\
Uno \emph{unit test} è tipicamente realizzato attraverso una classe al cui interno sono presenti diversi metodi, detti \emph{test cases}.\\
Vi sono delle precise regole da seguire per la creazione di uno unit test:
\begin{itemize}
\item Ogni metodo di test deve avere visibilità \texttt{public} e tipo di ritorno \texttt{void};
\item Ogni metodo di test deve essere annotato con una delle annotazioni fornite da JUnit (la più tipica è \texttt{@Test});
\item La valutazione del risultato deve essere fatta attraverso i metodi della classe \texttt{junit.framework.Assert}, la quale fornisce diverse tipologie di asserzioni.
\end{itemize}

In JUnit 4 (il più utilizzato al momento della scrittura di questa tesi) tutti i metodi di test vengono eseguiti in maniera sequenziale, ma in un ordine non meglio precisato.\\
Da JUnit 5 è stata invece introdotta la possibilità di eseguire più metodi di test in parallelo tra loro.\\

\subsubsection{Annotazioni principali}

Le annotazioni principali fornite da JUnit sono:\\
\begin{itemize}
\item \textbf{@Test}\\Dichiara che ciò che segue è un metodo di test. Ogni metodo preceduto da questa annotazione verrà testato a run-time.\\
\item \textbf{@Before}\\Dichiara che il metodo che segue deve essere eseguito prima di ogni esecuzione di un caso di test (tipicamente per re-inizializzare i parametri prima di ogni test).\\
\item \textbf{@After}\\Dichiara che il metodo che segue deve essere eseguito dopo ogni esecuzione di un caso di test (tipicamente per rilasciare le risorse precedentemente allocate).\\
\item \textbf{@BeforeClass}\\Dichiara che il metodo che segue sarà necessariamente statico, e verrà eseguito una sola volta prima di eseguire il primo test case della classe (tipicamente per inizializzare dati globali).\\
\item \textbf{@AfterClass}\\Dichiara che il metodo che segue sarà necessariamente statico, e verrà eseguito una sola volta dopo aver eseguito l'ultimo test case della classe (tipicamente per effettuare operazioni finali una volta ottenuti tutti i risultati dei test, come ad esempio un report su file).\\
\end{itemize}

\subsubsection{Asserzioni principali}

Le asserzioni principali fornite da JUnit sono:\\
\begin{itemize}
\item public static void \textbf{assertTrue}(boolean condition)\\Asserisce che \texttt{condition} sia valutata a \texttt{true}.\\
\item public static void \textbf{assertFalse}(boolean condition)\\Asserisce che \texttt{condition} sia valutata a \texttt{false}.\\
\item public static void \textbf{assertEquals}(Object expected, Object actual)\\Asserisce che \texttt{expected} sia valutato uguale a \texttt{actual}.\\
\item public static void \textbf{assertNull}(Object obj)\\Asserisce che \texttt{obj} sia valutato a \texttt{null}.\\
\end{itemize}

\subsubsection{Gestione delle eccezioni}
Altro aspetto da considerare è la gestione delle eccezioni, diversa a seconda che si utilizzi JUnit 4 o JUnit 5.\\

In JUnit 4 è necessario aggiungere, di seguito alla direttiva \texttt{@Test}, l'eccezione che si presuppone venga sollevata a causa dell'esecuzione del metodo, tramite la seguente sintassi:
\begin{verbatim}
@Test(expected = MyException.class)

\end{verbatim}

In JUnit 5 viene introdotta invece l'asserzione \texttt{assertThrows} che permette un controllo molto più preciso sulle eccezioni, in quanto è possibile testarne più di una in un unico test case.\\
La sintassi, che fa uso delle lambda-expressions di Java 8, è la seguente:
\begin{verbatim}
assertThrows(MyException.class, () -> doOperation(),
            "Expected doOperation() to throw MyException,
             but it didn't");
\end{verbatim}

\subsubsection{Un esempio completo}
Di seguito viene presentato un esempio di struttura completa di uno unit test, prendendo come esempio un'ipotetica classe \texttt{Razionale} che implementa appunto i numeri razionali.\\

\begin{lstlisting}
public class TestRazionale {
	//...
	
	@Test
	public void test1() {
		Razionale a = new Razionale(2, 3); // crea il numero razionale 2/3
		Razionale b = new Razionale(4, 6);
		assertTrue(a.equals(b));
	}
	
	@Test
	public void test2() {
		Razionale a = new Razionale(1, 2);
		assertNotNull(a);
	}
	
	//... 
}
\end{lstlisting}

\newpage
\thispagestyle{empty}
\mbox{}

\end{document}
